\documentclass{aux}

\title{D3} %Titre du fichier

\begin{document}

%----------- Informations du rapport ---------

\titol{Simplex Primal} %Titol del fitxer
\UE{FIB UPC} %Nom de la UE
\materia{Informe sobre la creació d'un solver de Simplex primal} %Nom de la materia

\tutor{ Linares, \textsc{MªPaz}} %Nom del tutor

\alumnes{
        Álvarez Aragones, \textsc{Ruben} \\
        Mejía Rota, \textsc{César Elías}}
         %Nom dels alumnes

%----------- Initialisation -------------------
        
\fairemarges %Mostrar margens
\fairepagedegarde
\tabledematieres
\section{Introduction}
This report contains all the exercices regarding Lab 3 of the CAP course.
\section{Module Environment}
This Task covers the primary usage of the module environment. Do the following
activities:
\subsection{Check the loaded modules}
\textsc{Check the loaded modules in your environment now using the correct command
for modules. Describe the modules that the system has loaded and made available}\\
\\
\textbf{Solution:}\\
To check the loaded modules in our environment, we use the command \textbf{module list}. This command shows the modules that are currently loaded in our environment.\\
The modules we find here are:
\begin{itemize}
    \item \textbf{Current Loaded Modules:}
    \begin{itemize}
        \item \textbf{1.} impi/2021.10.0
        \begin{itemize}
            \item a
        \end{itemize}
        \item \textbf{2.} mkl/2023.2.0
        \begin{itemize}
            \item  a                                                             
        \end{itemize}
        \item \textbf{3.} ucx/1.15.0
        \begin{itemize}
            \item   a                                                            
        \end{itemize}
        \item \textbf{4.} oneapi/2023.2.0
        \begin{itemize}
            \item    a                                                           
        \end{itemize}
        \item \textbf{5.} bsc/1.0
        \begin{itemize}
            \item     a                                                          
        \end{itemize}
        \item \textbf{6.} intel/2023.2.0
        \begin{itemize}
            \item      a                                                         
        \end{itemize}
        
    \end{itemize}
\end{itemize}

\subsection{Unload all modules}
\textsc{Unload all modules}\\
\\
\textbf{Solution:}\\
To unload all modules, we use the command \textbf{module purge}. This command unloads all the modules that are currently loaded in our environment.\\

\subsection{Load module intel/2023.2.0}
\textsc{Load the module intel/2023.2.0}\\
\\
\textbf{Solution:}\\
To load the module intel/2023.2.0, we use the command \textbf{module load intel/2023.2.0}. This command loads the module intel/2023.2.0 in our environment.\\

\subsection{Discover information about the specific Intel module version 2023.2.0}
\textsc{Discover information about the specific Intel module version 2023.2.0, including details on environment variables to be set, changes to the PATH, and other adjustments that will be applied when this module is loaded. Hint: “module
show"}\\

\textbf{Solution:}\\
To discover information about the specific Intel module version 2023.2.0, we use the command \textbf{module show intel/2023.2.0}. This command shows the information about the module intel/2023.2.0.\\
The information we find here is:

\begin{itemize}







\end{document}